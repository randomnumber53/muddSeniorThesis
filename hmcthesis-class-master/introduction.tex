\chapter{Introduction}

The central question of this thesis---How many qudit Bell states can a projective LELM device reliably distinguish?---is of both practical and theoretical importance. 





Measurements in the Bell basis are necessary for many quantum information protocols, such as quantum teleportation and superdense coding. In fact, being able to reliably distinguish sets of maximally entangled vectors (of which Bell bases are canonical examples)

\cite{coding}






% Structure of the Thesis

In Chapter 2 provides background for understanding the problem of LELM distinguishability. This includes information about the basics of quantum information, entanglement, 

Chapter 3 outlines relevant progress made towards 

In particular, it focuses on 1) group for d=3 and 2) hyperentangled case.


Chapter 4 presents the major results, providing a generalization of the group of transformations discussed in Chapter 3.\footnote{In particular, we generalize the group to higher $d$s.}


In Chapter 5, we discuss attempts to utilize information about the hyperentangled Bell basis 

