\chapter{Introduction}

The central question of this thesis---How many qudit Bell states can a projective LELM device reliably distinguish?---is of both practical and theoretical importance. 

Measurements in the Bell basis are necessary for many quantum information protocols, such as quantum teleportation and dense coding. In fact, each protocol for being able to reliably distinguish sets of maximally entangled vectors (of which Bell bases are canonical examples) has a one-to-one correspondence with both a quantum teleportation scheme as well as a dense coding scheme \cite{coding}. It is, however, difficult in practice to reliably make measurements in the Bell basis. This is because it is difficult to evolve one particle conditionally based on the first. 

For example, if the particles in question are photons that store information in their polarization, performing a measurement in the Bell basis would require the ability to alter the polarization of one photon based on the polarization of the other photon. While possible to do, it cannot be done reliably (i.e. with high probability of success).\footnote{Since dense coding schemes are meant to maximize the amount of information transferable by as few particles as possible, needing many copies of a particle to reliably produce an intended result is counterproductive.}

% Structure of the Thesis

The structure of this thesis is as such. Chapter 2 provides background for understanding the problem of LELM distinguishability. This includes information about the basics of quantum information, entanglement, and a formal definition of the LELM distinguishability problem. Chapter 3 outlines relevant progress made towards answering the LELM distinguishability problem. In particular, it focuses on: 1) a group distinguishability-preserving transformations for inputs in the $d=3$ case and 2) the LELM distinguishability problem for particles 'hyperentangled' in two variables.

Chapter 4 presents the major results, providing a generalization of the group of transformations discussed in Chapter 3.\footnote{In particular, we generalize the group to higher dimensional particles.} In Chapter 5, we discuss attempts to utilize information about the hyperentangled Bell basis to provide insight into the Bell basis LELM distinguish question when $d=4$.

% TODO: add motivation for quidits

